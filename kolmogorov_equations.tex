\documentclass[12pt,a4paper]{article}
\usepackage[utf8]{inputenc}
\usepackage[russian]{babel}
\usepackage{amsmath,amssymb,amsfonts}
\usepackage{graphicx}
\usepackage{hyperref}
\usepackage{xcolor}

\title{Уравнения Колмогорова для установившегося режима в задании 2}
\author{Непрерывные марковские цепи}
\date{\today}

\begin{document}

\maketitle

\section{Введение}

В данном документе рассматривается формирование матрицы интенсивностей переходов $Q$ и уравнения Колмогорова для установившегося режима работы ремонтируемой системы из задания 2.

\section{Описание системы}

Система состоит из устройств типа A и типа B с интенсивностями отказов $\lambda_A$ и $\lambda_B$ соответственно. Для функционирования системы требуется хотя бы одно устройство типа A и хотя бы $N_B$ устройств типа B. Также имеются резервные устройства в количествах $R_A$ и $R_B$ соответственно.

В системе возможна организация ремонта ранее вышедших из строя устройств. Одновременно может ремонтироваться только одно устройство. Если подлежат ремонту устройства разных типов, приоритет отдаётся тем, которых сломалось больше, а если их сломалось одинаковое число – тому типу, интенсивность поломок которого выше. Интенсивность ремонта устройств обоих типов одинакова и равна $\lambda_S$.

\section{Состояния системы}

Каждое состояние системы описывается тройкой $(a, b, r)$, где:
\begin{itemize}
    \item $a$ - количество работающих устройств типа A
    \item $b$ - количество работающих устройств типа B
    \item $r$ - статус ремонта:
    \begin{itemize}
        \item $r = 0$ - нет ремонта
        \item $r = 1$ - ремонт устройства типа A
        \item $r = 2$ - ремонт устройства типа B
    \end{itemize}
\end{itemize}

Общее количество состояний системы:
\begin{equation}
    N = (N_A + R_A + 1) \cdot (N_B + R_B + 1) \cdot 3
\end{equation}

\section{Матрица интенсивностей переходов $Q$}

Матрица $Q$ размера $N \times N$ содержит интенсивности переходов между всеми возможными состояниями системы. Элемент $Q_{ij}$ представляет интенсивность перехода из состояния $i$ в состояние $j$.

\subsection{Переходы, связанные с отказами устройств}

\subsubsection{Отказы устройств типа A}

Переход из состояния $(a, b, r)$ в состояние $(a-1, b, r')$:
\begin{itemize}
    \item Интенсивность: $a \cdot \lambda_A$
    \item Если $r = 0$ (нет ремонта), то $r' = 1$ (начинается ремонт A)
    \item Если $r \neq 0$, то $r' = r$ (продолжается текущий ремонт)
\end{itemize}

\subsubsection{Отказы устройств типа B}

Переход из состояния $(a, b, r)$ в состояние $(a, b-1, r')$:
\begin{itemize}
    \item Интенсивность: $b \cdot \lambda_B$
    \item Если $r = 0$ (нет ремонта), то $r' = 2$ (начинается ремонт B)
    \item Если $r \neq 0$, то $r' = r$ (продолжается текущий ремонт)
\end{itemize}

\subsection{Переходы, связанные с ремонтом устройств}

Переход из состояния $(a, b, r)$ в состояние $(a', b', r')$:
\begin{itemize}
    \item Интенсивность: $\lambda_S$
    \item Если $r = 1$ (ремонт A), то $a' = a+1$, $b' = b$
    \item Если $r = 2$ (ремонт B), то $a' = a$, $b' = b+1$
    \item $r'$ определяется приоритетом ремонта:
    \begin{itemize}
        \item Если больше сломанных устройств A: $r' = 1$
        \item Если больше сломанных устройств B: $r' = 2$
        \item Если поровну, выбирается по интенсивности отказов: $r' = 1$ если $\lambda_A \geq \lambda_B$, иначе $r' = 2$
        \item Если нет сломанных устройств: $r' = 0$
    \end{itemize}
\end{itemize}

\subsection{Диагональные элементы}

Диагональные элементы матрицы $Q$ содержат отрицательные суммы всех интенсивностей выхода из соответствующих состояний:
\begin{equation}
    Q_{ii} = -\sum_{j \neq i} Q_{ij}
\end{equation}

\section{Уравнения Колмогорова для установившегося режима}

В установившемся режиме вероятности состояний не меняются со временем, поэтому производные равны нулю:
\begin{equation}
    \frac{dp(t)}{dt} = 0
\end{equation}

Это приводит к системе линейных алгебраических уравнений:
\begin{equation}
    Q^T \cdot \pi = 0
\end{equation}

где:
\begin{itemize}
    \item $Q^T$ - транспонированная матрица интенсивностей переходов
    \item $\pi$ - вектор стационарных вероятностей
    \item $0$ - вектор нулей
\end{itemize}

Дополнительно необходимо условие нормировки:
\begin{equation}
    \sum_{i=0}^{N-1} \pi_i = 1
\end{equation}

\subsection{Уравнения баланса}

Для каждого состояния $i$ уравнение баланса имеет вид:
\begin{equation}
    \sum_{j \neq i} \lambda_{ji} \cdot \pi_j = \sum_{j \neq i} \lambda_{ij} \cdot \pi_i
\end{equation}

Что означает: сумма потоков вероятности, входящих в состояние $i$, равна сумме потоков вероятности, выходящих из состояния $i$.

\section{Решение системы уравнений}

Для решения системы уравнений Колмогорова для установившегося режима используется следующий алгоритм:

\begin{enumerate}
    \item Создается матрица $A = Q^T$
    \item Последняя строка $A$ заменяется на условие нормировки (все элементы = 1)
    \item Создается вектор правой части $b = [0, 0, \ldots, 1]^T$
    \item Решается система $A \cdot \pi = b$ методом QR-разложения
\end{enumerate}

Полученный вектор $\pi$ содержит стационарные вероятности всех состояний системы.

\section{Характеристики системы}

На основе стационарных вероятностей состояний можно вычислить следующие характеристики системы:

\subsection{Вероятность отказа системы}

Вероятность отказа системы равна сумме вероятностей состояний, в которых система не функционирует:
\begin{equation}
    P_{failure} = \sum_{a < 1 \text{ или } b < N_B} \pi_{(a,b,r)}
\end{equation}

\subsection{Среднее число готовых устройств}

Среднее число готовых устройств типа A:
\begin{equation}
    \bar{A} = \sum_{a,b,r} a \cdot \pi_{(a,b,r)}
\end{equation}

Среднее число готовых устройств типа B:
\begin{equation}
    \bar{B} = \sum_{a,b,r} b \cdot \pi_{(a,b,r)}
\end{equation}

\subsection{Коэффициент загрузки ремонтной службы}

Коэффициент загрузки ремонтной службы равен сумме вероятностей состояний, в которых происходит ремонт:
\begin{equation}
    K_{repair} = \sum_{r > 0} \pi_{(a,b,r)}
\end{equation}


\end{document}