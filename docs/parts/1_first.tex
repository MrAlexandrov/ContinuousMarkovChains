\subsection{Задача №1: Система без ремонта}\label{sec:First}

\subsubsection{Описание системы}

Система состоит из устройств типа A и типа B с интенсивностями отказов $\lambda_A$ и $\lambda_B$ соответственно. Для функционирования системы требуется хотя бы одно устройство типа A и хотя бы $N_B$ устройств типа B. Также имеются резервные устройства в количествах $R_A$ и $R_B$ соответственно, причём в нормальном состоянии одновременно включены сразу $N_A$ устройств типа A.

Параметры системы:
\begin{align}
\lambda_A &= G + (N \bmod 3) = 1 + (260 \bmod 3) = 1 + 2 = 3 \\
\lambda_B &= G + (N \bmod 5) = 1 + (260 \bmod 5) = 1 + 0 = 1 \\
N_A &= 2 + (G \bmod 2) = 2 + (1 \bmod 2) = 2 + 1 = 3 \\
N_B &= 1 + (N \bmod 2) = 1 + (260 \bmod 2) = 1 + 0 = 1 \\
R_A &= 1 + (G \bmod 2) = 1 + (1 \bmod 2) = 1 + 1 = 2 \\
R_B &= 2 - (G \bmod 2) = 2 - (1 \bmod 2) = 2 - 1 = 1
\end{align}

\subsubsection{Граф состояний системы}

Состояние системы определяется парой $(a, b)$, где $a$ - количество работающих устройств типа A, $b$ - количество работающих устройств типа B. Система работоспособна, если $a \geq 1$ и $b \geq N_B = 1$.

Общее количество состояний системы:
\begin{equation}
(N_A + R_A + 1) \cdot (N_B + R_B + 1) = (3 + 2 + 1) \cdot (1 + 1 + 1) = 6 \cdot 3 = 18
\end{equation}

Граф состояний системы представлен на рисунке \texttt{state\_graph\_task1.png}.

\subsubsection{Матрица интенсивностей переходов}

Матрица интенсивностей переходов $Q$ размера $18 \times 18$ содержит интенсивности переходов между всеми возможными состояниями системы. Элемент $Q_{ij}$ представляет интенсивность перехода из состояния $i$ в состояние $j$.

Переходы в системе происходят при отказах устройств:
\begin{itemize}
    \item Отказ устройства типа A: переход из состояния $(a, b)$ в состояние $(a-1, b)$ с интенсивностью $a \cdot \lambda_A$
    \item Отказ устройства типа B: переход из состояния $(a, b)$ в состояние $(a, b-1)$ с интенсивностью $b \cdot \lambda_B$
\end{itemize}

Диагональные элементы матрицы $Q$ содержат отрицательные суммы всех интенсивностей выхода из соответствующих состояний:
\begin{equation}
Q_{ii} = -\sum_{j \neq i} Q_{ij}
\end{equation}

Граф переходов системы представлен на рисунке \texttt{transition\_graph\_task1.png}.

\subsubsection{Дифференциальные уравнения Колмогорова}

Дифференциальные уравнения Колмогорова для вероятностей состояний системы имеют вид:
\begin{equation}
\frac{dp(t)}{dt} = Q^T \cdot p(t)
\end{equation}

где:
\begin{itemize}
    \item $p(t)$ - вектор вероятностей состояний системы в момент времени $t$
    \item $Q^T$ - транспонированная матрица интенсивностей переходов
\end{itemize}

Начальное условие: $p(0) = [1, 0, \ldots, 0]^T$, что соответствует состоянию, когда все устройства исправны.

\subsubsection{Решение уравнений Колмогорова}

Для решения системы дифференциальных уравнений Колмогорова используется метод Рунге-Кутты 4-го порядка. Результаты решения представлены на графике вероятностей состояний системы (\texttt{states\_probabilities\_task1.png}).

\subsubsection{Функция надёжности системы}

Функция надёжности системы $R(t)$ определяется как вероятность того, что система работоспособна в момент времени $t$:
\begin{equation}
R(t) = \sum_{a \geq 1, b \geq N_B} p_{(a,b)}(t)
\end{equation}

График функции надёжности представлен на рисунке \texttt{reliability\_function\_task1.png}.

\subsubsection{Математическое ожидание времени безотказной работы}

Математическое ожидание времени безотказной работы (MTTF) вычисляется как интеграл от функции надёжности:
\begin{equation}
MTTF = \int_{0}^{\infty} R(t) dt
\end{equation}

В программе этот интеграл вычисляется численно методом трапеций:
\begin{equation}
MTTF \approx \sum_{i=1}^{n} \frac{R(t_{i-1}) + R(t_i)}{2} \cdot (t_i - t_{i-1})
\end{equation}

Полученное значение MTTF (аналитическое): $0.736181$

\subsubsection{Имитационное моделирование}

Имитационное моделирование системы в терминах непрерывных марковских цепей проводится 100 раз. Для каждого эксперимента моделируется случайная траектория системы до момента отказа.

Алгоритм моделирования:
\begin{enumerate}
    \item Начальное состояние: $a = N_A + R_A = 5$, $b = N_B + R_B = 2$, $t = 0$
    \item Пока система работоспособна ($a \geq 1$ и $b \geq N_B = 1$):
    \begin{enumerate}
        \item Вычисляем суммарную интенсивность переходов: $\lambda_{total} = a \cdot \lambda_A + b \cdot \lambda_B$
        \item Генерируем время до следующего события: $\Delta t \sim Exp(\lambda_{total})$
        \item Обновляем время: $t = t + \Delta t$
        \item С вероятностью $\frac{a \cdot \lambda_A}{\lambda_{total}}$ происходит отказ устройства типа A: $a = a - 1$
        \item С вероятностью $\frac{b \cdot \lambda_B}{\lambda_{total}}$ происходит отказ устройства типа B: $b = b - 1$
    \end{enumerate}
    \item Возвращаем время до отказа системы $t$
\end{enumerate}

Результаты имитационного моделирования:
\begin{itemize}
    \item MTTF (имитационный): $0.58089$
    \item Стандартное отклонение: $0.323707$
\end{itemize}

\subsubsection{Гистограмма времени безотказной работы}

Для визуализации распределения времени безотказной работы системы необходимо построить гистограмму результатов имитационного моделирования. Функция \texttt{plotHistogram} объявлена в \texttt{GnuplotPlotter.h}, но не реализована в исходном коде.

Реализация этой функции должна создавать гистограмму на основе вектора времен отказов, полученных в результате имитационного моделирования. Примерный код реализации:

\begin{verbatim}
void GnuplotPlotter::plotHistogram(
    const std::vector<double>& data,
    const std::string& title,
    const std::string& outputPrefix
) {
    std::ofstream script(outputPrefix + ".gp");
    script << "set terminal png size 800,600\n";
    script << "set output '" << outputPrefix << ".png'\n";
    script << "set title '" << title << "'\n";
    script << "set xlabel 'Время до отказа'\n";
    script << "set ylabel 'Частота'\n";
    script << "set grid\n";

    // Определяем количество бинов по правилу Стёрджеса
    int numBins = 1 + 3.322 * log10(data.size());

    // Находим минимальное и максимальное значения
    double minVal = *std::min_element(data.begin(), data.end());
    double maxVal = *std::max_element(data.begin(), data.end());

    // Вычисляем ширину бина
    double binWidth = (maxVal - minVal) / numBins;

    script << "binwidth = " << binWidth << "\n";
    script << "bin(x,width) = width*floor(x/width)\n";
    script << "set boxwidth binwidth\n";

    // Сохраняем данные во временный файл
    std::ofstream datafile(outputPrefix + ".dat");
    for (double value : data) {
        datafile << value << "\n";
    }
    datafile.close();

    script << "plot '" << outputPrefix << ".dat' using (bin($1,binwidth)):(1.0) "
           << "smooth freq with boxes title 'Frequency'\n";

    script.close();

    std::string command = "gnuplot " + outputPrefix + ".gp";
    std::system(command.c_str());
}
\end{verbatim}

Для использования этой функции в \texttt{main.cpp} необходимо добавить код:

\begin{verbatim}
// Получаем времена отказов из 100 экспериментов
std::vector<double> failureTimes;
failureTimes.reserve(100);
for (int i = 0; i < 100; ++i) {
    failureTimes.push_back(simulator.simulateSingleRun());
}

// Строим гистограмму
GnuplotPlotter::plotHistogram(failureTimes, "Распределение времени безотказной работы", "failure_times_histogram_task1");
\end{verbatim}

\newpage
