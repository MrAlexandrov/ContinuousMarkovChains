\subsection*{Задание}\label{sec:Task}
\addcontentsline{toc}{subsection}{Задание}

Система состоит из устройств типа $A$ и типа $B$, интенсивности отказов $\lambda_A$ и $\lambda_B$ известны.
Для функционирования системы требуется хотя бы одно устройство типа $A$ и хотя бы $N_B$
устройств типа $B$. Также имеются резервные устройства в количествах $R_A$ и $R_B$
соответственно, причём в нормальном состоянии одновременно включены сразу $N_A$
устройств типа $A$.

Если $N$ – номер зачётной книжки, а $G$ – последняя цифра в номере группы, то параметры
системы определяются следующим образом ($N = 260$, $G = 1$):

\begin{itemize}
        \item $\lambda_A$     = $G$ + ($N$ mod 3)     = 3
        \item $\lambda_B$     = $G$ + ($N$ mod 5)     = 1
        \item $N_A$           = 2 + ($G$ mod 2)       = 3
        \item $N_B$           = 1 + ($N$ mod 2)       = 1
        \item $R_A$           = 1 + ($G$ mod 2)       = 2
        \item $R_B$           = 2 – ($G$ mod 2)       = 1
\end{itemize}

Требуется:

\begin{enumerate}
        \item нарисовать граф состояний системы;
        \item составить матрицу интенсивностей переходов;
        \item записать дифференциальные уравнения Колмогорова;
        \item методами численного интегрирования решить полученную систему дифференциальных уравнений, исходя из того, что в начальный момент времени все устройства исправны;
        \item построить графики вероятностей нахождения системы в каждом из возможных состояний с течением времени;
        \item построить график функции надёжности системы;
        \item рассчитать математическое ожидание времени безотказной работы;
        \item провести имитационное моделирование системы в терминах непрерывных марковских цепей 100 раз, рассчитать среднее выборочное значение и стандартное отклонение времени безотказной работы системы.
\end{enumerate}

\newpage

\newpage
