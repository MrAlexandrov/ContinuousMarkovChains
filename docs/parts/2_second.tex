\subsection{Задача №2: Система с ремонтом}\label{sec:Second}

\subsubsection{Описание системы}

Рассматривается система, аналогичная задаче №1, но в которой возможна организация ремонта ранее вышедших из строя устройств. Одновременно может ремонтироваться только одно устройство. Если подлежат ремонту устройства разных типов, приоритет отдаётся тем, которых сломалось больше, а если их сломалось одинаковое число – тому типу, интенсивность поломок которого выше.

Параметры системы:
\begin{align}
\lambda_A &= 3 \\
\lambda_B &= 1 \\
N_A &= 3 \\
N_B &= 1 \\
R_A &= 2 \\
R_B &= 1 \\
\lambda_S &= 3
\end{align}

\subsubsection{Граф состояний и матрица переходов}

Состояние системы определяется парой $(a, b)$, где $a$ - количество работающих устройств типа A, $b$ - количество работающих устройств типа B. Система работоспособна, если $a \geq 1$ и $b \geq N_B = 1$.

Общее количество состояний системы:
\begin{equation}
(N_A + R_A + 1) \cdot (N_B + R_B + 1) = (3 + 2 + 1) \cdot (1 + 1 + 1) = 6 \cdot 3 = 18
\end{equation}

Граф состояний системы представлен на рисунке \texttt{state\_graph\_task2.png}.

Матрица интенсивностей переходов $Q$ размера $18 \times 18$ содержит интенсивности переходов между всеми возможными состояниями системы. Элемент $Q_{ij}$ представляет интенсивность перехода из состояния $i$ в состояние $j$.

Переходы в системе происходят при отказах и ремонте устройств:
\begin{itemize}
    \item Отказ устройства типа A: переход из состояния $(a, b)$ в состояние $(a-1, b)$ с интенсивностью $a \cdot \lambda_A$
    \item Отказ устройства типа B: переход из состояния $(a, b)$ в состояние $(a, b-1)$ с интенсивностью $b \cdot \lambda_B$
    \item Ремонт устройства типа A: переход из состояния $(a, b)$ в состояние $(a+1, b)$ с интенсивностью $\lambda_S$, если ремонтируется устройство типа A
    \item Ремонт устройства типа B: переход из состояния $(a, b)$ в состояние $(a, b+1)$ с интенсивностью $\lambda_S$, если ремонтируется устройство типа B
\end{itemize}

Приоритет ремонта определяется следующим образом:
\begin{itemize}
    \item Если количество неисправных устройств типа A больше, чем типа B, то ремонтируется устройство типа A
    \item Если количество неисправных устройств типа B больше, чем типа A, то ремонтируется устройство типа B
    \item Если количество неисправных устройств обоих типов одинаково, то ремонтируется устройство того типа, интенсивность отказов которого выше (в нашем случае, типа A, так как $\lambda_A > \lambda_B$)
\end{itemize}

\subsubsection{Алгебраические уравнения Колмогорова для установившегося режима}

В установившемся режиме вероятности состояний не меняются со временем, поэтому производные равны нулю:
\begin{equation}
\frac{dp_i(t)}{dt} = 0, \quad i = 0, 1, \ldots, 17
\end{equation}

Это приводит к системе линейных алгебраических уравнений. Для каждого состояния $i$ уравнение имеет вид:
\begin{equation}
\sum_{j \neq i} q_{ji} \cdot \pi_j = \sum_{j \neq i} q_{ij} \cdot \pi_i
\end{equation}

где $q_{ij}$ - элемент матрицы интенсивностей переходов $Q$, $\pi_i$ - стационарная вероятность состояния $i$.

Запишем эти уравнения в матричной форме:
\begin{equation}
Q^T \cdot \pi = 0
\end{equation}

где $Q^T$ - транспонированная матрица интенсивностей переходов, $\pi$ - вектор стационарных вероятностей, $0$ - вектор нулей.

Дополнительно необходимо условие нормировки:
\begin{equation}
\sum_{i=0}^{17} \pi_i = 1
\end{equation}

Рассмотрим несколько примеров уравнений для конкретных состояний:

Для состояния $(5, 2)$ (все устройства исправны):
\begin{align}
\frac{dp_{(5,2)}(t)}{dt} &= 0 \\
-\pi_{(5,2)} \cdot (5\lambda_A + 2\lambda_B) + \pi_{(4,2)} \cdot \lambda_S + \pi_{(5,1)} \cdot \lambda_S &= 0 \\
-\pi_{(5,2)} \cdot (5 \cdot 3 + 2 \cdot 1) + \pi_{(4,2)} \cdot 3 + \pi_{(5,1)} \cdot 3 &= 0 \\
-\pi_{(5,2)} \cdot 17 + \pi_{(4,2)} \cdot 3 + \pi_{(5,1)} \cdot 3 &= 0
\end{align}

Для состояния $(4, 2)$ (одно устройство типа A неисправно):
\begin{align}
\frac{dp_{(4,2)}(t)}{dt} &= 0 \\
\pi_{(5,2)} \cdot 5\lambda_A - \pi_{(4,2)} \cdot (4\lambda_A + 2\lambda_B + \lambda_S) + \pi_{(3,2)} \cdot \lambda_S + \pi_{(4,1)} \cdot \lambda_S &= 0 \\
\pi_{(5,2)} \cdot 15 - \pi_{(4,2)} \cdot (12 + 2 + 3) + \pi_{(3,2)} \cdot 3 + \pi_{(4,1)} \cdot 3 &= 0 \\
\pi_{(5,2)} \cdot 15 - \pi_{(4,2)} \cdot 17 + \pi_{(3,2)} \cdot 3 + \pi_{(4,1)} \cdot 3 &= 0
\end{align}

Для состояния $(5, 1)$ (одно устройство типа B неисправно):
\begin{align}
\frac{dp_{(5,1)}(t)}{dt} &= 0 \\
\pi_{(5,2)} \cdot 2\lambda_B - \pi_{(5,1)} \cdot (5\lambda_A + \lambda_B + \lambda_S) + \pi_{(4,1)} \cdot \lambda_S + \pi_{(5,0)} \cdot \lambda_S &= 0 \\
\pi_{(5,2)} \cdot 2 - \pi_{(5,1)} \cdot (15 + 1 + 3) + \pi_{(4,1)} \cdot 3 + \pi_{(5,0)} \cdot 3 &= 0 \\
\pi_{(5,2)} \cdot 2 - \pi_{(5,1)} \cdot 19 + \pi_{(4,1)} \cdot 3 + \pi_{(5,0)} \cdot 3 &= 0
\end{align}

Аналогичные уравнения можно записать для всех остальных состояний системы.

\subsubsection{Предельные вероятности состояний системы}

Для решения системы уравнений Колмогорова для установившегося режима используется следующий алгоритм:
\begin{enumerate}
    \item Создается матрица $A = Q^T$
    \item Последняя строка $A$ заменяется на условие нормировки (все элементы = 1)
    \item Создается вектор правой части $b = [0, 0, \ldots, 1]^T$
    \item Решается система $A \cdot \pi = b$ методом QR-разложения
\end{enumerate}

\subsubsection{Характеристики системы}

На основе стационарных вероятностей состояний вычисляются следующие характеристики системы:

\paragraph{Вероятность отказа системы}
Вероятность отказа системы равна сумме вероятностей состояний, в которых система не функционирует:
\begin{equation}
P_{failure} = \sum_{a < 1 \text{ или } b < N_B} \pi_{(a,b)}
\end{equation}

Полученное значение: $0.545037$

\paragraph{Среднее число готовых устройств}
Среднее число готовых устройств типа A:
\begin{equation}
\bar{A} = \sum_{a,b} a \cdot \pi_{(a,b)} = 1.90617
\end{equation}

Среднее число готовых устройств типа B:
\begin{equation}
\bar{B} = \sum_{a,b} b \cdot \pi_{(a,b)} = 0.83057
\end{equation}

\paragraph{Коэффициент загрузки ремонтной службы}
Коэффициент загрузки ремонтной службы равен сумме вероятностей состояний, в которых происходит ремонт:
\begin{equation}
K_{repair} = \sum_{(a,b) \neq (N_A+R_A, N_B+R_B)} \pi_{(a,b)} = 0.993679
\end{equation}

\subsubsection{Дифференциальные уравнения Колмогорова}

Дифференциальные уравнения Колмогорова для вероятностей состояний системы имеют вид:
\begin{equation}
\frac{dp(t)}{dt} = Q^T \cdot p(t)
\end{equation}

где $p(t)$ - вектор вероятностей состояний системы в момент времени $t$, $Q^T$ - транспонированная матрица интенсивностей переходов.

Запишем эти уравнения для нескольких состояний:

Для состояния $(5, 2)$ (все устройства исправны):
\begin{align}
\frac{dp_{(5,2)}(t)}{dt} &= -p_{(5,2)}(t) \cdot (5\lambda_A + 2\lambda_B) + p_{(4,2)}(t) \cdot \lambda_S + p_{(5,1)}(t) \cdot \lambda_S \\
&= -p_{(5,2)}(t) \cdot 17 + p_{(4,2)}(t) \cdot 3 + p_{(5,1)}(t) \cdot 3
\end{align}

Для состояния $(4, 2)$ (одно устройство типа A неисправно):
\begin{align}
\frac{dp_{(4,2)}(t)}{dt} &= p_{(5,2)}(t) \cdot 5\lambda_A - p_{(4,2)}(t) \cdot (4\lambda_A + 2\lambda_B + \lambda_S) + p_{(3,2)}(t) \cdot \lambda_S + p_{(4,1)}(t) \cdot \lambda_S \\
&= p_{(5,2)}(t) \cdot 15 - p_{(4,2)}(t) \cdot 17 + p_{(3,2)}(t) \cdot 3 + p_{(4,1)}(t) \cdot 3
\end{align}

Для состояния $(5, 1)$ (одно устройство типа B неисправно):
\begin{align}
\frac{dp_{(5,1)}(t)}{dt} &= p_{(5,2)}(t) \cdot 2\lambda_B - p_{(5,1)}(t) \cdot (5\lambda_A + \lambda_B + \lambda_S) + p_{(4,1)}(t) \cdot \lambda_S + p_{(5,0)}(t) \cdot \lambda_S \\
&= p_{(5,2)}(t) \cdot 2 - p_{(5,1)}(t) \cdot 19 + p_{(4,1)}(t) \cdot 3 + p_{(5,0)}(t) \cdot 3
\end{align}

Начальное условие: $p(0) = [1, 0, \ldots, 0]^T$, что соответствует состоянию, когда все устройства исправны.

\subsubsection{Решение уравнений Колмогорова и время переходного процесса}

Для решения системы дифференциальных уравнений Колмогорова используется метод Рунге-Кутты 4-го порядка. 

Время переходного процесса оценивается как время, необходимое для того, чтобы эвклидова норма вектора невязки с ранее рассчитанным предельным вектором составляла не более 1\% эвклидовой нормы последнего:
\begin{equation}
\|p(t) - \pi\| \leq 0.01 \cdot \|\pi\|
\end{equation}

Оценка времени переходного процесса: $6.00601$

Время моделирования выбирается вдвое больше теоретической оценки времени переходного процесса: $2 \cdot 6.00601 = 12.01202$

Результаты решения представлены на графике вероятностей состояний системы (\texttt{states\_probabilities\_task2.png}).

\subsubsection{Имитационное моделирование}

\paragraph{Моделирование в терминах непрерывных марковских цепей}
Результаты моделирования:
\begin{itemize}
    \item Вероятность отказа системы: $0.929348$
    \item Среднее число готовых устройств типа A: $1.345498$
    \item Среднее число готовых устройств типа B: $0.132479$
    \item Коэффициент загрузки ремонтной службы: $0.978996$
\end{itemize}

График траектории марковского процесса представлен на рисунке \texttt{repairable\_markov\_chain\_trajectory.png}.

\paragraph{Моделирование в терминах дискретно-событийного моделирования}
Результаты моделирования:
\begin{itemize}
    \item Вероятность отказа системы: $0.386486$
    \item Среднее число готовых устройств типа A: $0.960225$
    \item Среднее число готовых устройств типа B: $1.020829$
    \item Коэффициент загрузки ремонтной службы: $0.993894$
\end{itemize}

График траектории дискретно-событийного процесса представлен на рисунке \texttt{repairable\_discrete\_event\_trajectory.png}.

\newpage
