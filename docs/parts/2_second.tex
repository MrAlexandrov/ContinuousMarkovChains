\subsection{Анализ надёжности системы}\label{sec:reliability}

\subsubsection{Основные положения теории надёжности}
Основной характеристикой надёжности системы является вероятность безотказной работы $P(t)$, которая описывается экспоненциальным законом распределения:

\begin{equation}
P(t) = e^{-\lambda t}
\end{equation}

где $\lambda$ -- интенсивность отказов системы, $t$ -- время работы.

\subsubsection{Марковская модель системы}
Для анализа надёжности сложных систем используем марковские процессы с непрерывным временем. Состояния системы представлены на рис.\ \ref{fig:state_graph}.

\begin{figure}[H]
\centering
\includegraphics[width=0.8\textwidth]{images/state_graph.png}
\caption{Граф состояний системы}
\label{fig:state_graph}
\end{figure}

Система дифференциальных уравнений Колмогорова:

\begin{equation}
\begin{cases}
\frac{dP_0}{dt} = -(\lambda_{01} + \lambda_{02})P_0(t) + \mu_{10}P_1(t) + \mu_{20}P_2(t) \\
\frac{dP_1}{dt} = \lambda_{01}P_0(t) - (\mu_{10} + \lambda_{13})P_1(t) \\
\frac{dP_2}{dt} = \lambda_{02}P_0(t) - (\mu_{20} + \lambda_{23})P_2(t) \\
\frac{dP_3}{dt} = \lambda_{13}P_1(t) + \lambda_{23}P_2(t)
\end{cases}
\end{equation}

\subsubsection{Результаты расчётов}
На рис.\ \ref{fig:reliability_function} представлены графики вероятностей состояний системы при различных значениях интенсивностей отказов.

\begin{figure}[H]
\centering
\includegraphics[width=0.95\textwidth]{images/reliability_function.png}
\caption{Вероятности состояний системы во времени}
\label{fig:reliability_function}
\end{figure}

Основные выводы:
\begin{itemize}
\item Среднее время безотказной работы системы составляет $T_{ср} = 1/\lambda = 850$ ч
\item Вероятность безотказной работы за 1000 часов: $P(1000) = 0.318$
\item Критическими компонентами являются блоки 1 и 2 (см. рис.\ \ref{fig:state_graph})
\end{itemize}
